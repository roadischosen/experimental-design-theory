\documentclass{article}
\usepackage{cmap}
\usepackage[utf8]{inputenc}
\usepackage[english,ukrainian]{babel}
\usepackage[a4paper, mag=1000, left=2.5cm, right=1cm, top=2cm, bottom=1cm, headsep=0.7cm]{geometry}
\usepackage{multirow}
\usepackage{graphicx}
\usepackage{wrapfig}
\usepackage{listings}
\usepackage[T1]{fontenc}
\usepackage{PTMono}

\pagenumbering{gobble}

\begin{document}
    \begin{center}

        \large НАЦІОНАЛЬНИЙ ТЕХНІЧНИЙ УНІВЕРСИТЕТ УКРАЇНИ
        «КИЇВСЬКИЙ ПОЛІТЕХНІЧНИЙ ІНСТИТУТ» ІМ.ІГОРЯ СІКОРСЬКОГО\\
        ФАКУЛЬТЕТ ІНФОРМАТИКИ І ОБЧИСЛЮВАЛЬНОЇ ТЕХНІКИ
        КАФЕДРА ОБЧИСЛЮВАЛЬНОЇ ТЕХНІКИ \\[5.5cm]
    
        \huge Лабораторна робота №2 \\[0.6cm]
        \large з дисципліни «Теорія планування експерименту»\\
        \large на тему: «»\\[3.5cm]
    
    \end{center}
    
    \begin{flushright}
        \large
        Виконали:\\
        студенти 3-го курсу гр. IO-53 \\
        Котіков Нікіта \\
        Земiн Володимир \\
        ЗК №5315, 5310\\
        Перевірив:\\
        Виноградов Ю.I.
    \end{flushright}
    
    
    \vfill
    
    \begin{center}
        \large Київ --- \the\year р.
    \end{center}
    
    \newpage
    \large
    \section*{Мета:}
    Провести двофакторний експеримент, перевірити однорідність дисперсії за
    критерієм Романовського, отримати коефіцієнти рівняння регресії, провести натуралізацію
    рівняння регресії.
    
    \section*{Завдання:}
    Варiант: 313\\
    \begin{tabular}{|l|l|l|l|l|}
        \hline
        \multirow{2}{*}{№ Вар.} & \multicolumn{2}{l|}{$x_1$} & \multicolumn{2}{l|}{$x_2$}                         \\ \cline{2-5}
                                & $x_{min}$                  & $x_{max}$                  & $x_{min}$ & $x_{max}$ \\ \hline
        313                     & -15                        & 30                         & -35       & 15        \\ \hline
    \end{tabular}
    
	\lstset{
	  basicstyle=\ttfamily,
	  columns=fullflexible,
	  breaklines=true,
	  postbreak=\mbox{$\hookrightarrow$}\space,
	}
    \section*{Лістинг програми:}
    \texttt{\small \lstinputlisting[language={Python}]{lab2.py}}

    \section*{Матриця планування і результати:}

    \begin{tabular} {|c|c|c|c|c|}
        \hline
        № & $\overline{x_1}$ & $\overline{x_2}$ & $\overline{y}$ & $y_{теор}$ \\ \hline
        1 &               -1 &               -1 &          133.5 &     133.50 \\ \hline
        2 &               -1 &               +1 &          101.0 &     101.00 \\ \hline
        3 &               +1 &               -1 &           93.5 &      93.50 \\ \hline
    \end{tabular}
    \\\\
    Нормалізоване рівняння: $y = 97.25 - 20x_1 - 16.25x_2$ \\
    Натуралізоване рівняння: $y = 97.42 - 0.89x_1 - 0.65x_2$
    \section*{Висновок:}
    Були вивченi основні поняття, визначення, принципи теорії планування експерименту, написано програму для проведення двофакторний експеримент, підібору кількісті дослідів, достатньої для однорідності дисперсії, розраховання коефіцієнти регресії і її натуралізації.

\end{document}